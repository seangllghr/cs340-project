---
author: Sean Gallagher
title: "Milestone Two: RESTful Web Service"
short-title: "RESTful Web Service"
running-head: "RESTful Web Service"
course: "CS-340: Client/Server Development"
university: Southern New Hampshire University
header-block: true
title-page: false
bibliography: refs.bib
listings: true
linestretch: 1.8
---

\vskip -12pt
The specification for the second milestone called for the implementation of the
actual REST API code, including API endpoints and the code required to transform
requests to these API endpoints into calls to the data access layer created in
Milestone 1. In an attempt to approach this project in as professional a manner
as possible---and lacking any authoritative consesus as to the best approach to
the problem---I've selected an architecture based roughly on that described by
@clearyProjectStructureExpress2020. While this architecture adds additional
complexity to the project that is likely unnecessary at this level, the
additional learning experience seemed worth the effort. For ease of
testing, all of the relevant `curl` commands were combined into a single shell
script, interspersed with additional read requests and text to make the
resultant output presentation-ready. The full script, `curls.sh`, is listed in
Listing \ref{lst:curls}. Figure \ref{fig:curl-results} shows the results of
executing `curls.sh`.

\lstset{
  backgroundcolor=\color{white},
  breakatwhitespace=false,
  breaklines=false,
  captionpos=b,
  columns=fullflexible,
  commentstyle=\color{mediumgray}\upshape,
  emph={},
  emphstyle=\color{crimson},
  extendedchars=true,  % requires inputenc
  fontadjust=true,
  frame=single,
  identifierstyle=\color{black},
  keepspaces=true,
  keywordstyle=\color{mediumblue},
  keywordstyle={[2]\color{darkviolet}},
  keywordstyle={[3]\color{royalblue}},
  numbers=left,
  numbersep=5pt,
  numberstyle=\tiny\color{black},
  rulecolor=\color{black},
  showlines=true,
  showspaces=false,
  showstringspaces=false,
  showtabs=false,
  stringstyle=\color{forestgreen},
  tabsize=2,
  title=\lstname,
  upquote=true  % requires textcomp
}

\begin{lstlisting}[
language=sh,
caption={The Bash script used to generate output verifying correct operation of
  the four API endpoints.},
label={lst:curls}
]
#!/usr/bin/env bash

echo "Testing Create API Endpoint:"
curl \
    -H "Content-Type: application/json" \
    -X POST \
    -d @create.json \
    'http://localhost:3000/api/create'
echo ""

echo "Testing Read API Endpoint:"
curl 'http://localhost:3000/api/read?business_name=ACME+TEST+INC.'
echo ""

echo "Testing Update API Endpoint:"
curl 'http://localhost:3000/api/update?id=10011-2017-TEST&result=Violation+Issued'
echo ""

echo "Verification:"
curl 'http://localhost:3000/api/read?business_name=ACME+TEST+INC.'
echo ""

echo "Testing Delete API Endpoint:"
curl 'http://localhost:3000/api/delete?id=10011-2017-TEST'
echo ""

echo "Verification:"
curl 'http://localhost:3000/api/read?business_name=ACME+TEST+INC.'
echo ""
\end{lstlisting}

\begin{figure}[htbp]
  \includegraphics[width=\textwidth]{./build/img/curl-results.png}
  \caption{Results of running \texttt{curls.sh} against the running Express
    application.}
  \label{fig:curl-results}
\end{figure}

\pagebreak

# References
